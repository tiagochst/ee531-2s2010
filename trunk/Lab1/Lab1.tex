%-%-%-%-%-%-%-%-%-%-%-%-%-%-%-%-%-%-%-%-%-%-%-%-%-%
% EE531: laboratório de Eletrônica Básica I       %  
% Experimento 1: Familizarização com instrumentos %
%                de medida                        %
% Data:07/08/2010                                 %
% Unicamp,Campinas,São Paulo,Brasil               % 
% Grupo:                                          %
%       - Raquel Mayumi Kawamoto                  %
%       - Tiago Chedraoui Silva                   % 
%-%-%-%-%-%-%-%-%-%-%-%-%-%-%-%-%-%-%-%-%-%-%-%-%-%
%\documentclass[letter]{article}  % formato impressao IC
\documentclass[a4paper]{article} % formato impressao FEEC

%%% fontes %%%
\usepackage[T1]{fontenc}
\usepackage[brazil]{babel}    % dá suporte para os termos na língua portuguesa do Brasi
\usepackage[utf8]{inputenc}   % acentuação


%%% outros %%%
\usepackage{textcomp}
\usepackage{color}       
\usepackage{indentfirst}      % retira padrao americano de paragrafos
\usepackage{multicol}   
\usepackage[linkbordercolor={1 1 1},urlcolor=black,colorlinks=true]{hyperref} % links


% Capa estilizada %
\newcommand*{\titleTMB}{\begingroup \centering \settowidth{\unitlength}{\LARGE EE531} {\large\scshape EE531 - Turma S}\\[0.2\baselineskip] \rule{11.0cm}{1.6pt}\vspace*{-\baselineskip}\vspace*{2pt} \rule{11.0cm}{0.4pt}\\[\baselineskip] {\LARGE  Familizarização com instrumentos de medida}\vspace*{\baselineskip}  {\itshape Laboratório de Eletrônica Básica I - Segundo Semestre de 2010}\\ \rule{11.0cm}{0.4pt}\vspace*{-\baselineskip}\vspace{3.2pt} \rule{11.0cm}{1.6pt}\\[\baselineskip] {\large\scshape Professor: José Cândido Silveira Santos Filho}\par \vfill {\normalsize   \scshape 
    \begin{center} 
      \begin{tabular}{  l  l  p{5cm} } 
        Raquel Mayumi Kawamoto & RA: 086003\\
        Tiago Chedraoui Silva  & RA: 082941\\
      \end{tabular} \end{center}
    \itshape \today }\\[\baselineskip] \vspace{3.2pt} \endgroup}


\begin{document}
\titleTMB 
\newpage
\section{Dados experimentais}
Os dados obtidos da tabela \ref{tab:cursors} e da tabela \ref{tab:measure}...

%Amplitude pico-à-pico 9.8 V
%período 100 micros
%tempo de subida
%tempo de descida 40 micro segundos 
%offset +1 v+ +2
%          v- +2
%offset -1 v+ -2
%          v- -2

%Vrms 2.88 V
%Vavg -57.1 mV
%Vpp 9.92V
%Prd 100 micro
%Rise 42 micro seg
%Fall 42 micro seg

\begin{table}[h]
\begin{centering}
\begin{tabular}{cc}
\hline 
Descrição & Valor\tabularnewline
\hline
Amplitude pico-a-pico & 9.8\tabularnewline
Período & 100$\mu s$\tabularnewline
Tempo de subida & 40$\mu s$\tabularnewline
Tempo de descida  & 40$\mu s$\tabularnewline
\hline
\end{tabular}
\par\end{centering}

\caption{Dados experimentais obtidos através do recurso cursor \label{tab:cursors}}

\end{table}



\begin{table}[h]
\begin{centering}
\begin{tabular}{cc}
\hline 
Descrição & Valor\tabularnewline
\hline
Amplitude pico-a-pico & 9.92\tabularnewline
Período & 100$\mu s$\tabularnewline
Tempo de subida & 42$\mu s$\tabularnewline
Tempo de descida  & 42$\mu s$\tabularnewline
$V_{avg}$ & -57.1 mV\tabularnewline
$V_{rms}$ & 2.88v V \tabularnewline
\hline
\end{tabular}
\par\end{centering}

\caption{Dados experimentais obtidos através do recurso measure \label{tab:measure}}

\end{table}


\end{document}
